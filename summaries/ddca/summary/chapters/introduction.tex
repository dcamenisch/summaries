\chapter{Introduction}
\section{Description}

The class provides a first introduction to the design of digital circuits and 
computer architecture. It covers technical foundations of how a computing platform 
is designed from the bottom up. It introduces various execution paradigms, hardware 
description languages, and principles in digital design and computer architecture. 
The focus is on fundamental techniques employed in the design of modern microprocessors 
and their hardware/software interface.
\bigskip

\begin{satz}[Overview]
    This class provides a first approach to Computer Architecture. The students learn the
    design of digital circuits in order to:
    \begin{itemize}
        \item understand the basics
        \item understand the principles of design
        \item understand the precedents in computer architecture
    \end{itemize}
\tcblower
    Based on such understanding, the students are expected to:
    \begin{itemize}
        \item learn how a modern computer works underneath, from the bottom up
        \item evaluate tradeoffs of different designs and ideas
        \item implement a principled design (a simple microprocessor) 
        \item learn to systematically debug increasingly complex systems
        \item hopefully be prepared to develop novel, out-of-the-box designs
    \end{itemize}
    
\end{satz}
\bigskip

The focus is on basics, principles, precedents, and how to use them to 
create/implement good designs.