\chapter{Kreise}
\section{Eulertour}
\begin{definition}
    Sei $G = (V,E)$ ein Graph. Eine \textbf{Eulertour} ist ein geschlossener \text{Weg}, der jede Kante genau
    einmal enthält. Enthält ein Graph eine Eulertour, so nennt man ihn \textbf{eulersch}.
\end{definition}
\bigskip

Eine solche Tour lässt sich in $\mathcal{O}(|E|)$ finden - der Algorithmus (''schneller und langsamer Läufer'')
lässt sich ungefähr wie folgt beschreiben:

\begin{algorithm}
    \caption{Eulertour(G,s)}
    \begin{algorithmic}[1]
        \State $W \leftarrow$ \Call{Randomtour}{s} \Comment{''Läufer''}
        \State $v_{slow} \leftarrow $ Startknoten von $W$ \Comment{''Schildkröte''}
        \While{$v_{slow}$ ist nicht der letzte Knoten in $W$} 
            \State $v \leftarrow$ Nachfolger von $v_{slow}$ in $W$
            \If{$\exists (v,w) \in E, w \notin W$} \Comment{Ungenutze Kanten ab $v$}
                \State $W' \leftarrow$ \Call{Randomtour}{s}
                \State $W \leftarrow W_1 + W' + W_2$
            \EndIf
            \State $v_{slow} \leftarrow $ Nachfolger von $v_{slow}$ in $W$
        \EndWhile
        \Return $W$
    \end{algorithmic}
\end{algorithm}

\begin{algorithm}
    \caption{Randomtour(s)}
    \begin{algorithmic}[1]
        \State $v \leftarrow s$
        \State $W \leftarrow \{v\}$
        \While{$\exists (v,w) \in E, w \notin W$} \Comment{Ungenutze Kanten ab $v$}
            \State Wähle beliebigen Nachfolger $v_{next}$
            \State Hänge $v_{next}$ an $W$ analog
            \State $e \leftarrow \{v, v_{next}\}$
            \State Lösche $e$ aus $G$
            \State $v \leftarrow v_{next}$
        \EndWhile
        \State \Return $W$
    \end{algorithmic}
\end{algorithm}

Die Laufzeit ist leicht erklärt: Wir betrachten jede Kante genau einmal und ''löschen'' sie danach.

\begin{satz}[Satz]
    Ein zusammenhängender Graph $G = (V,E)$ enthält eine Eulertour $\Leftrightarrow$ der Grad jedes Knoten
    $v \in V$ ist gerade.
\end{satz}

\section{Hamiltonkreis}
\begin{definition}
    Sei $G = (V,E)$ ein Graph. Ein \textbf{Hamiltonkreis} ist ein Kreis, der alle Knoten von $V$ genau einmal
    durchläuft. Enthält ein Graph einen Hamiltonkreis, so nennt man ihn \textbf{hamiltonisch}.
\end{definition}
\bigskip

Ein klassisches Anwendungsbeispiel ist das Traveling Salesman Problem. Es wird vermutet, dass es keinen 
Algorithmus gibt, der in polynomieller Zeit bestimmt, ob es einen Hamiltonkreis in einem gegebenen Graphen 
gibt. \\

Es gibt aber einige Spezialfälle, für die es deutlich leichter ist, die Existenz eines Hamiltonkreises 
zu bestimmen:

\begin{itemize}
    \item Ein $n \times m$ Gitter enthält einen Hamiltonkreis genau dann wenn $n * m$ gerade ist
    \item Ein d-dimensionaler Hyperwürfel $H_d$ (Knotenmenge: $\{0, 1\}^d$, Kantenmenge: Alle Knotenpaare, welche sich in genau einer Koordinate unterscheiden) enthält einen Hamiltonkreis (auch für Dimensionen $d \geq 4$)
\end{itemize}

\begin{satz}[Satz von Dirac]
    Jeder Graph $G = (V, E)$ mit $|V| \geq 3$ und Minimalgrad $\delta(G) \geq |V| / 2$ enthält
    einen Hamiltonkreis. 
\end{satz}
\bigskip

Wenn wir einen Graphen anschauen, der nicht in einer dieser Kategorien fällt, wird es schwieriger um zu 
entscheiden, ob der Graph einen Hamiltonkreis enthält. Wenn wir das Problem mit Brute Force angehen, dauert
es $\mathcal{O}(n!)$. Nun wollen wir uns zwei Algorithmen anschauen, die etwas effizienter sind.

\pagebreak

\begin{algorithm}
    \caption{Hamiltonkreis(G)}
    \begin{algorithmic}[1]
        \For{$v \in [n]$ mit $v \neq 1$} \Comment{Initialisierung}
            \State $P_{(1,x), x} =
            \begin{cases}
                1, \text{ falls }(1,x) \in E\\
                0, \text{ sonst }
            \end{cases} $
        \EndFor

        \For{$s = 3$ bis $n$} \Comment{Initialisierung}
            \For{$S \subseteq [n]$ mit $1 \in S$ und $|S| = s$}
                \For{$x \in S, x \neq 1$}
                    \State $P_{s,x} = \text{max} \{ P_{S \backslash {x}, y}: y \in S \cap N(x), y \neq 1 \} $
                \EndFor
            \EndFor
        \EndFor

        \If{$\exists x \in N(1)$ mit $P_{[n],x} = 1$} \Comment{Ausgabe}
            \State \Return $G$ enthält einen Hamiltonkreis
        \Else
            \State \Return $G$ enthält keinen Hamiltonkreis
        \EndIf
    \end{algorithmic}
\end{algorithm}

Dieser Algorithmus kann in $\mathcal{O}(|V|^2 \cdot 2^{|V|})$ berechnen ob ein Hamiltonkreis existiert. Jedoch braucht
er dafür Speicherplatz in $\mathcal{O}(|V| \cdot 2^{|V|})$. Dies ist nicht optimal, daher gibt es einen zweiten Algorithmus,
der nicht nur die Existenz von Hamiltonkreisen feststellt, sondern auch gleich die Anzahl der Hamiltonkreise berechnet.

\begin{definition}
    $W_s = $ Anzahl der Wege der Länge $n$ von $s$ nach $s$ die keinen Knoten in $S \subset V$ besuchen. 
\end{definition}

\begin{algorithm}
    \caption{Zähle-Hamiltonkreis(G)}
    \begin{algorithmic}[1]
        \State $s = 1$ \Comment{Wahl des Startknoten (willkürlich)}
        \State $Z = |W_\emptyset|$
        \For{}
            \State Berechne $|W_S|$ \Comment{mit der Adjazenzmatrix von $G[V\backslash S]$}
            \State $Z = Z + (-1)^{|S|} \cdot |W_S|$
        \EndFor
        \State $Z = Z / 2$
        \State \Return $Z$ \Comment{Anzahl der Hamiltonkreise in $G$}
    \end{algorithmic}
\end{algorithm}

Dieser zweite Algorithmus hat eine Laufzeit von $\mathcal{O}(|V|^{2.81} \cdot \text{log}(|V|) \cdot 2^{|V|})$, braucht
dafür aber nur Speicherplatz in $\mathcal{O}(|V|^2)$, weshalb dieser Algorithmus in der Praxis meist besser ist. \\

Wir bemerken den drastischen Laufzeitunterschied gegenüber der Eulertour: Dort benötigten wir gerade 
einmal $\mathcal{O}(|E|)$ Zeit, um eine solche zu finden, während das Hamiltonkreisproblem nicht in polynomieller Zeit lösbar ist, da es 
NP-vollständig ist.

\section{NP-Probleme}

\begin{definition}
    Ein NP-Problem, ist ein Problem, dass nicht in polynomieller Laufzeit gelöst werden kann 
    (man nimmt es mindestens an). Das Gegenteil dazu sind P-Probleme.
\end{definition}

\section{Traveling Salesman}

Das Traveling Salesman Problem ist ein sehr berühmtes Problem:

\begin{definition}
    \textit{Gegeben sei ein vollständiger Graph mit n Knoten, berechne die kürzeste Rundreise.}
\end{definition}
\bigskip

Dies bedeutet effektiv, dass wir den minimalen Hamiltonkreis suchen. Das TSP ist stark mit dem Hamiltonkreisproblem
verwandt, aber wir haben nun statt einer binären Antwort, viele Antwortmöglichkeiten. Wir können aber aus dem TSP
ein Optimierungsproblem machen. Bezeichned opt$(K_n,l)$ die optimale Lösung für einen Graphen $K_n$ mit
Gewichtsfunktion $l$, so sprechen wir von einem $\alpha$-Approximationsalgorithmus, falls dieser immer eine 
Lösung $C$ mit
$$\sum_{e \in C} l(e) \leq \alpha \cdot \text{opt}(K_n, l)$$
findet. Allerdings sind wir so immernoch sehr eingeschränkt, denn wenn wir einen Graphen so konstruieren, dass
wir das Hamiltonkreisproblem lösen : Alle Kanten, die in einem anderen Graph (der zu untersuchen ist) vorkommen,
haben Gewicht 0. Da $\alpha \cdot 0 = 0$ müsste ein $\alpha$-Approximation immer einen Hamiltonkreis in seiner
Laufzeit finden. \\

Mithilfe einer zusätzlichen Annahme, können wir das TSP in aktzeptabler Laufzeit approximieren: Wir erwarten,
dass die Kantenlänge die \textit{Dreiecksungleichung} erfüllen, dass sich also Umwege nie lohnen würden. Wir
nenne dies, dass metrische TSP.

\begin{satz}[Satz]
    Für das metrische TSP gibt es einen 2-Approximationsalgorithmus mit Laufzeit $\mathcal{O}(n^2)$.
\end{satz}
\bigskip

Wir erreichen dies wie folgt: Wir berechne zuerst den MST des Graphen in $\mathcal{O}(n^2)$. Nun verdoppeln wir
alle Kanten - dadurch erhalten wir einen Weg von Länge $2 \cdot$ MST. Nun finden wir eine Eulertour, und durchlaufen
diese erneut, wobei wir 'Abkürzungen' verwenden, also keinen bereits besuchten Knoten erneut besuchen. Aufgrund der
Dreiecksungleichung wird unser Weg dadruch garantiert nicht länger. Da aus einem Hamiltonkreis durch entfernen einer
Kante ein Spannbaum wird, muss offensichtlich gelten dass
$$\text{MST}(K_n,l) \leq \text{opt}(K_n,l)$$
Insbesondere muss also die Länge unseres gefundenen Hamiltonkreises kleiner sein als $2 \cdot \text{MST}$, und damit kleiner
als $2 \cdot \text{opt}$, was ja genau zu zeigen war.